Throughout development, it is essential to gather continuous feedback from both users and stakeholders. Each time a new feature (e.g., AI-driven internship suggestions, improved matching algorithms) is implemented, we incorporate feedback to refine the system further.

During an alpha test, a select group of users—often representatives from universities or pilot companies—try out the platform in a controlled environment. Their impressions help us measure the platform’s ease of use and identify initial flaws. It can be particularly valuable to involve individuals who understand the internship process deeply, such as career services staff or HR personnel, as they can provide specialized feedback on whether S\&C meets real-world needs.

The alpha test is also instrumental in identifying malfunctions or major design issues before moving on to the beta test, in which a broader set of real users—students from multiple institutions and additional companies—access the platform under more realistic conditions. This phase helps detect remaining bugs and usability issues in what approximates a production environment.

Finally, once S\&C is deployed, continuous monitoring remains crucial. Certain logs and usage statistics (e.g., large spikes in the number of applications or error rates in authentication) should be automatically collected and made available to developers. These logs allow for ongoing debugging and performance tuning, ensuring that the platform remains reliable and efficient for all participants as usage grows and additional features are introduced.
