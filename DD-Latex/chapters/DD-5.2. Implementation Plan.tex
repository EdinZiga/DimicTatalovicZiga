The aim of this section is to describe the implementation strategies that will be used to develop, integrate, and test the various components of the Students & Companies application. The objective is to leverage the advantages of both the bottom-up and threads strategies.

Using a threads strategy is effective because it makes progress visible to users and stakeholders, allowing for early feedback and engagement. This approach also minimizes the need for placeholder components, although it does make the integration process more complex.

A top-down methodology will be applied by first designing a basic structure or framework for the application. More advanced features, such as internship matching, application management, and feedback systems, will then be added incrementally as individual threads once they are validated.

This strategy enables different development teams to work concurrently on distinct tasks. Once a feature is fully developed and tested, it will be integrated into the overall software architecture. This approach ensures that validated components contribute to building a robust and functional application while maintaining steady progress and stakeholder visibility.

\subsection{Features identification}