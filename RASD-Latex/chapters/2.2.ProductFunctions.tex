\quad \textbf{User Registration and Authentication} \\
The platform allows students, companies, and universities to register by creating accounts and logging in using secure authentication methods. Besides mentioned users, the platform has administrator account that is responsible for verification of the companies. Users can register by pressing the button "Sign up" and the platform will send them on the registration page where they will enter necessary data. Students and universities have an option to sign up using an institution account, while companies after the registration have to go through verification process by an administrator of the platform. Students provide their CV and skills, companies enter relevant business information, and universities add institutional details. This ensures that all users are identified and role-specific functionalities are enabled. If the user is already registered on the platform, there is an option to login using user credentials. \\

\textbf{Profile Management} \\
Each user can manage their profiles by updating personal information. The platform offers the page with the possibility of viewing data of the logged in user and modifying it. Students can upload and update their CVs (by filling the textboxs related to the sections of the CV; e.g. work experience, education, skills) and preferences; companies can refine their organizational descriptions; and universities can manage their institutional profiles. This enables users to keep their information relevant and up-to-date. \\

\textbf{Internship Advertisement Creation and Management} \\
Companies are able to access the page for advertisement creation by pressing the button "Create new internship advertisement" where they can create internship advertisements by providing detailed information about the role, required skills, offered benefits, and the selection process. In addition, there is an option to view all active internship advertisements offered by that company, where they can be updated, activated, deactivated, or deleted to ensure only relevant opportunities are visible to students. \\

\textbf{Internship Search and Application} \\
The platform offers students possibility to search for internships using filters like location, required skills, and benefits. They can view detailed descriptions of each internship by pressing on the box where the advertisement is located. After pressing on the selected advertisement the platform sends user on the page where details of the internship are presented. Button "Apply" is located on the internship page. That way, student can submit application directly through the platform. The number of applications per student is unlimited. \\

\textbf{Selection Process Management} \\
On the company user page on the platform, the company can view the status of any student that is in the selection process. The company can distribute among the candidates and view the answers, schedule an interview with the student using an option "Schedule with Google Meets", view the date and time of scheduled interview and also can terminate the selection process anytime if decides that the student is not right fit for the internship role, with giving the reason for such decision. On the other hand, the student that is in the active selection process can view their progress, view the questionnaire and the deadline for submitting it. As the company, also the student has an option to view the date and time of the scheduled interview with the selected company. \\

\textbf{Feedback System} \\
Feedback system which is integrated on the platform is responsible for collecting information from the students and companies after the selection process finishes and after the internships conclude. Companies provide feedback on the students’ performance and their experience using the platform, while students share their thoughts about the platform or about the internship and the company. In order to avoid any bias if the student is admitted to the internship, reviews are anonymous. Data collected this way is used for improvement of the platform and in order to build trust between users. \\

\textbf{Recommendation System} \\
The recommendation system is built with the intention to suggest internships to students based on their CVs, skills, and preferences. Method that the recommendation system uses is  a keyword-matching algorithm to identify opportunities that align with a student’s profile, helping them find relevant internships more efficiently. On the home page of the platform when the logged in user is student where the internship opportunities are shown, the first one which are showed are the ones that recommendation system chose. Those internship advertisements have label \textit{Suggested} in order to highlight them.\\

\textbf{Complaint Handling System} \\
Both students and companies can submit complaints about issues that arise during selection process or internships that are active or completed. The platform provides an option on the page of the active internship or selection process to file a complaint. Complaints are reviewed by universities, which act as mediators to resolve disputes. When the complaint is requested by the student or the company, the university account gets a notification about the active complaint request. This system ensures that the platform maintains ethical and fair practices. \\
%Add how is the complaint handled by the university??

\textbf{Internship Monitoring}\\
Universities are able to monitor ongoing internships by accessing feedback and progress reports. Progress reports are requested for the student and the company every month. This function ensures compliance with academic and ethical standards. This way universities can monitor the internship progress and assign points to students for completed internships when required. \\

\textbf{Notifications System} \\
The platform is equipped with the notification system which sends notifications to students and companies at various stages, such as when new internships are posted, application statuses are updated, or when students are invited to the next step in the selection process. Notifications keep users informed and engaged. \\

