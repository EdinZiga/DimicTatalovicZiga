% To define the scope of the product we can use ``The World  \&  Machine'' approach by M. Jackson and P. Zave.
% We can define the real world entities that interact with the system (the World), the entities that belong to the system (the Machine) and the shared phenomena (the intersection of the two other sets).


The Students \& Companies platform is designed to streamline the internship process by addressing the needs of three main user groups: Companies, Students, and Universities. Each group is provided with custom functionalities to support their respective roles in the internship ecosystem. \\

The platform enables companies to efficiently create and manage internship advertisements. These advertisements include detailed descriptions of requirements, projects, and relevant terms, ensuring clarity for prospective applicants. Companies can accept internship applications from students and handle the entire selection process, supported by tools for conducting interviews and distributing structured questionnaires. While the platform is conducting all the necessary steps in order to complete the selection process, interviews are supposed to be held on the external service (Microsoft Teams, Google Meets...) Once an internship is filled, companies are encouraged to provide feedback on their experience with the platform and its functionalities. \\

Students can utilize the platform to upload their data and CVs, creating comprehensive profiles that highlight their skills and experiences. They can search for internship opportunities that align with their goals and proactively initiate the application process. The platform enables easy and smooth participation in the selection process by allowing students to respond to invitations to interviews and questionnaires. In addition, students receive timely notifications regarding their selection status and are encouraged to provide feedback on their experience. For any challenges encountered during an internship, the platform provides a mechanism for students to submit complaints. \\

Universities play a crucial role in ensuring the integrity of the internship process by handling complaints submitted by students. They monitor and resolve issues, ensuring that internships provide a safe and productive environment. This oversight contributes to maintaining the reputation of the platform as a trusted intermediary between students and companies. Handling of the complaints should be done between the university, student, and company directly and not through the platform.

\subsubsection{World Phenomena}

\begin{itemize}
    \item \textbf{WP1} Students have their own CVs, experiences, and preferences when looking for internships.
    \item \textbf{WP2} Companies create internship opportunities and search for qualified candidates for their roles.
    \item \textbf{WP3} Company employees decide how the process of selection will look like before accessing the platform.
    \item \textbf{WP4} Complaints and conflicts can arise during internships, requiring resolution by a neutral entity (universities).
    \item \textbf{WP5?} Students participate in interviews and selection processes conducted by companies.
    
\end{itemize}

\subsubsection{Shared Phenomena}

\textbf{World controlled}
\begin{itemize}
    \item \textbf{SPW1} Students upload their CVs, experiences, projects and personal data.
    \item \textbf{SPW2} Companies interact with the system to upload internship advertisements and define the process of selection.
    \item \textbf{SPW3} Students use the system to search for internships and submit applications.
    \item \textbf{SPW4} Students and companies exchange information through the platform, including interview schedules and questionnaires.
    \item \textbf{SPW5} Companies evaluate students who went through the selection process and mark the selected ones.
    \item \textbf{SPW6} Students that got accepted decide whether they want to enroll the selected internship
    \item \textbf{SPW7} Universities use the system to access complaints submitted by students and monitor internship progress.
    
\end{itemize}
\textbf{Machine controlled}
\begin{itemize}    
    \item \textbf{SPM1} The system recommends internships to students based on their CVs, projects and preferences
    \item \textbf{SPM2} Notifications about new internships or application statuses are automatically sent to users.
    \item \textbf{SPM3} The system requests for feedback and ratings from both students and companies, contributing to platform analytics and recommendations.
    \item \textbf{SPM4} The platform provides suggestions to students and companies to improve their CVs or internship postings, based on statistical and user feedback.
    \item \textbf{SPM5} The system tracks and records complaints submitted by students and sends them to universities.
\end{itemize}

\subsubsection{Machine Phenomena}
\begin{itemize}
    \item \textbf{MP1} The system stores and processes student profiles, including CVs and other personal data.
    \item \textbf{MP2} The system contains a searchable database of internship advertisements.
\end{itemize}


